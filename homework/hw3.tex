\documentclass[11pt]{article}

% set 1-inch margins in the document
\usepackage[margin=1in]{geometry}
\usepackage{amsthm}
\usepackage{amsmath}
\usepackage{amssymb,amsfonts}
\theoremstyle{definition}

% include this if you want to import graphics files with /includegraphics
\usepackage{graphicx}
\providecommand{\abs}[1]{\lvert#1\rvert}

% info for header block in upper right hand corner

\newtheorem{problem}{Problem}

\title{HIT --- Cryptography --- Homework 3}

\begin{document}

\maketitle

\begin{problem}
In our attack on a 1-round substitution-permutation network, we considered a block length of 64 bits and a network with 16 $S$-boxes that each take a 4-bit input. 
\begin{enumerate}
\item Repeat the analysis for the case of 8 $S$-boxes, each taking an 8-bit input. What is the complexity of the attack now?
\item Repeat the analysis again with a 128-bit block length and 16 $S$-boxes that each take an 8-bit input.
\item Does the $S$-boxes length make any difference? Does the block length make any difference?
\end{enumerate}
\end{problem}

\begin{problem}
Show that DES has the property that $DES_k(x) = \overline{DES_{\overline{k}}(\overline{x})}$ for every key $k$ and input $x$ (where $\overline{z}$ denotes the bitwise complement of $z$). This is called the complementarity property of $DES$.
\end{problem}

\begin{problem}
Is the addition function $f(x, y) = x + y$ (where $|x| = |y|$ and $x$ and $y$ are interpreted as natural numbers) a one-way function?
\end{problem}

\begin{problem}
Let $f_{1}(x)$ and $f_{2}(x)$ be one-way functions. Is $f(x) = (f_{1}(x), f_{2}(x))$ necessarily a one-way function? Prove your answers.
\end{problem}

\begin{problem}
Let $f$ be a one-way function. Is $g(x) = f(f(x))$ necessarily a one-way function? What about $g(x) = (f(x),f(f(x)))$? Prove your answers.
\end{problem}

\end{document}